\documentclass[a4paper, 12pt]{report}
\usepackage{monapack}

\begin{document}

	\begin{titlepage}
		\begin{center}
			\includegraphics[width=3cm]{logo}\\[2mm]
			Střední průmyslová škola a Vyšší odborná škola, Písek, Karla Čapka 402, Písek\\[2mm]
			%18-20-M/01 Informační technologie\\[2cm]
			%{\Huge \bf Manuál} \\[2mm]
			%\normalsize k maturitní práci téma č. 11\\[6mm]
			%{\textbf{\huge{Šablona DMP v \LaTeX u}}}\\[2mm]
			18-20-M/01 Informační technologie\\[1,2cm]
			\Huge Šablona DMP v \LaTeX u\\[6mm]
			{\textbf{\Huge{Manuál}}}\\[2mm]
			\normalsize k maturitní práci téma č. 11
		\end{center}
		\vfill
		\noindent autor:\\
		\emph{\bf Monika Hanušová, B4.I}\\[2mm]
		vedoucí maturitní práce:\\
		\emph{\bf Mgr. Radka Pecková}\\[8mm]
		Písek 2017/2018
	\end{titlepage}

	\obsah

	\kapitola{Struktura šablony}
	Příkaz \strojopis{$\backslash$documentclass[a4paper, 12pt]\{report\}} na začátku dokumentu slouží k určení všeobecných charakteristik tzv. tříd. Nepovinný parametr (v hranatých závorkách) optimalizuje sazbu na velikost papíru A4 a základní velikost písma stanovuje na 12 bodů. Povinný parametr (ve složených závorkách) specifikuje třídu dokumentu report, která je určena pro přípravu rozsáhlejších dokumentů.\\
	Příkaz \strojopis{$\backslash$usepackage\{monapack\}} zajišťuje zavedení balíku (stylu) vytvořeného pro psaní DPM. Tento balík je nazván \kurziva{monapack} a obsahuje zdrojový text makrojazyka \TeX u.\\
	Pomocí následujících příkazů se definují základní informace o autorovi a o maturitní práci. Tyto hodnoty jsou následně využity při generaci titulní strany, licenční smlouvy i zadání.\\
	\\
	\strojopis{
		$\backslash$student$\{$Jméno Příjmení$\}$\\
		$\backslash$trida$\{$Třída$\}$\\
		$\backslash$obor$\{$Kód oboru Název oboru$\}$\\
		$\backslash$bydliste$\{$Adresa bydliště$\}$\\
		$\backslash$datumNarozeni$\{$Datum narození$\}$\\
		$\backslash$vedouci$\{$Titul, jméno, příjmení$\}$\\
		$\backslash$nazevPrace$\{$Název práce$\}$\\
		$\backslash$cisloPrace$\{$Číslo$\}$\\
		$\backslash$skolniRok$\{$Školní rok$\}$\\
		$\backslash$reditel$\{$Titul, jméno a příjmení$\}$\\
	}
	\novastrana 
	Praktický příklad:\\
	\strojopis{
		$\backslash$student$\{$Monika Hanušová$\}$\\
		$\backslash$trida$\{$B4.I$\}$\\
		$\backslash$obor$\{$18-20-M/01 Informační technologie$\}$\\
		$\backslash$bydliste$\{$tř. Čsl. legií 703/16, 370 06 České Budějovice$\}$\\
		$\backslash$datumNarozeni$\{$10. 2. 1998$\}$\\
		$\backslash$vedouci$\{$Mgr. Radka Pecková$\}$\\
		$\backslash$nazevPrace$\{$Šablona DMP v \LaTeX u$\}$\\
		$\backslash$cisloPrace$\{$11$\}$\\
		$\backslash$skolniRok$\{$2017/2018$\}$\\
		$\backslash$reditel$\{$Ing. Jiří Uhlík$\}$\\
	}
	\\
	Součástí šablony jsou předepsané příkazy, jež v sobě obsahují nadefinované instrukce pro vzhled stránky a další informace. Příkazy \strojopis{$\backslash$zacatek} a \strojopis{$\backslash$konec} určují začátek a konec dokumentace. \\
	Pro vygenerování titulní stránky se použije příkaz \strojopis{$\backslash$titulniStrana} a licenční smlouva se vygenerujete příkazem \strojopis{$\backslash$licencniSmlouva\{Datum odevzdání\}}. První parametr „Datum odevzdání“ určuje datum, kdy je fyzicky odevzdávána maturitní práce.\\
	Za každý z následujících příkazů se vždy umístí příslušný text:\\
	\strojopis{
		$\backslash$anotace Vlastní text anotace\\
		$\backslash$annotation Vlastní text anotace v anglickém jazyce\\
		$\backslash$podekovani Vlastní text poděkování\\
	}
	Příkaz \strojopis{$\backslash$obsah} vygeneruje obsah, \strojopis{$\backslash$seznamTabulek} vygeneruje seznam tabulek a seznam obrázků se vygeneruje pomocí \strojopis{$\backslash$seznamObrazku}. Tyto příkazy již neobsahují žádné povinné ani nepovinné parametry. O tvorbě příloh a literatury se dozvíte v dalších kapitolách, viz kapitola 4.10 a~4.11.
	
	\kapitola{Zadání}
	Zadání maturitní práce se vygeneruje příkazem \strojopis{$\backslash$\{termín odevzdání\}\{zadání\}\\
		\{originalita a vhodnost řešení\}\{funkčnost řešení\}\{hrazení nákladů\}\\
		\{práce je majetkem\}\{datum podepsání\}}.\\
	Parametr 2 až 4 jsou bodové seznamy, jež se vytvoří pomocí příkazu \strojopis{$\backslash$bod} a vložením textu bezprostředně za tento příkaz. Pokračuje se stejným způsobem se všemi body. Pátý parametr dokončuje větu „Náklady na materiál bude hradit…“. Do parametru se vepíše jedna z těchto variant: škola/firma/žák. Šestý parametr dokončuje větu: „Funkční vzorek bude majetkem…“. Do parametru se použije jedna z těchto variant: školy/firmy/žáka.\\
	Zadání je vytvořeno i jako samostatný soubor \kurziva{LaTeX\_task} z důvodu potřeby samostatného souboru již dříve než je psána samotná dokumentace. Po vytisknutí dokumentace je vhodné vyměnit část obsahující zadání za již potvrzené zadání. 
	
	\kapitola{Kapitoly a podkapitoly}
	K rozdělení textu do jednotlivých úrovní slouží kapitoly a podkapitoly. Jsou připraveny celkem čtyři úrovně, jež lze využít:\\
	\strojopis{
		$\backslash$kapitola\{název kapitoly aneb nadpis 1. úrovně\}\\
		$\backslash$podkapitola\{název podkapitoly aneb nadpis 2. úrovně\}\\	
		$\backslash$podpodkapitola\{název podpodkapitoly aneb nadpis 3. úrovně\}\\
		$\backslash$podpodpodkapitola\{název podpodpodkapitoly aneb nadpis 4. úrovně\}\\
	}
	\\
	Praktický příklad:\\
	\strojopis{
		$\backslash$kapitola\{Úvod\}\\
		$\backslash$podkapitola\{Téma\}\\
		$\backslash$podpodkapitola\{Řešení\}\\
		$\backslash$podpodpodkapitola\{Problematika\}\\
	}\\
	Po vysázení: (\viz{sazba01})\\
	\obrazek{sazba01}{Kapitoly}{6cm}{sazba01}
	
	\kapitola{Psaní textu}
	\podkapitola{Ukončení odstavce}
	Pro ukončení odstavce je třeba na konec textu přidat dvě zpětná lomítka \strojopis{$\backslash \backslash$}. Další text již bude umístěn na nové řádce.
	
	\podkapitola{Nová strana}
	Šablona sama rozmisťuje strany dle normy. Je-li potřeba i tak vynutit novou stranu, použije se následující příkaz \strojopis{$\backslash$novastrana}.
	
	\podkapitola{Nedělitelná mezera}
	Na konci řádku by se dle typografických pravidel nemělo objevit jedno písmeno či předložka. Používáme proto nedělitelnou mezeru. Ta se v sázecím prostředí \LaTeX ~píše tildou neboli vlnovkou $\sim$ (pravý alt + +).
	
	\podkapitola{Uvozovky}
	Pro umístění textu mezi uvozovky se použije příkaz \strojopis{$\backslash$uv\{text v uvozovkách\}}.
	
	\podkapitola{Pomlčky}
	V \LaTeX u se rozlišuje krátká pomlčka tzv. spojovník, normální pomlčka, dlouhá pomlčka tzv. rozdělovník a matematické mínus, \viztab{pomlcky}.
	\tab{pomlcky}{Pomlčky}{|c|c|c|c|c|}{
		\cara
		Význam & Spojovník & Pomlčka & Rozdělovník & Mat. mínus\\
		\cara
		Zápis & - & - - & - - - & \$-\$\\
		\cara
	}
	
	\podkapitola{Speciální znaky}
	Pro psaní speciálních znaků v textu, jež by mohly ovlivnit kód, se před ně vloží zpětné lomítko \kurziva{$\backslash$}. Jedná se například o \#, \$ a další. 
	
	\podkapitola{Komentáře}
	Pro psaní komentáře se umístí na začátek řádku procento \kurziva{\%}. Text za procentem se poté nevysází do výsledného dokumentu.
	
	\podkapitola{Řezy písma}
	Pro zvýraznění textu lze využít různé řezy písma. Pro programové kódy, názvy složek a~další se hodí strojopis \strojopis{$\backslash$strojopis\{text\}} a pro kapitálky se použije příkaz \strojopis{$\backslash$kapitalky\\\{text\}}. Pro zdůraznění informace v textu lze vyzkoušet kurzívu \strojopis{$\backslash$kurziva\{text\}}, tučné písmo \strojopis{$\backslash$tucne\{text\}} nebo zvýraznění \strojopis{$\backslash$zvyraznit\{text\}}. Zvýraznění \LaTeX ~provádí zvolením dostatečně odlišného řezu písma. Je-li okolní text (nezvýrazněný) sázen patkově, pak je zvýrazněný text sázen kurzívou a podobně.
	
	\kapitola{Seznamy}
	
	\podkapitola{Číslovaný seznam}
	Číslovaný seznam se vytvoří příkazem \strojopis{$\backslash$cislseznam\{$\backslash$bod text $\backslash$bod text2…\}}. Pro jednotlivé položky seznamu se do povinného parametru (složených závorek) vloží \strojopis{$\backslash$bod}, za který se umístí text 1. bodu. Takto se pokračuje s libovolným počtem položek. \\
	\\
	Praktický příklad:\\
	\strojopis{
		$\backslash$cislseznam\{\\
		\indent $\backslash$bod základní deska\\
		\indent $\backslash$bod grafická karta\\
		\indent $\backslash$bod procesor\\
		\indent $\backslash$bod hard disk\\
		\indent $\backslash$bod zvuková karta\\
		\}\\[1cm]
	}
	Po vysázení:
	\cislseznam{
		\bod základní deska
		\bod grafická karta
		\bod procesor
		\bod hard disk
		\bod zvuková karta	
	}
	
	\podkapitola{Nečíslovaný seznam}
	Obdobným způsobem jako číslovaný seznam lze vytvořit nečíslovaný (bodový) seznam s~použitím příkazu \strojopis{$\backslash$bodseznam\{$\backslash$bod text $\backslash$bod text2\}}.\\
	Praktický příklad:\\
	\strojopis{
		$\backslash$bodseznam\{\\
		\indent $\backslash$bod základní deska\\
		\indent $\backslash$bod grafická karta\\
		\indent $\backslash$bod procesor\\
		\indent $\backslash$bod hard disk\\
		\indent $\backslash$bod zvuková karta\\
		\}\\[1cm]
	}
	Po vysázení:
	\bodseznam{
		\bod základní deska
		\bod grafická karta
		\bod procesor
		\bod hard disk
		\bod zvuková karta
	}
	
	\podkapitola{Popisový seznam}
	U popisového seznamu je místo bodu použita tzv. položka. Položka $\backslash$polozka se skládá ze dvou povinných parametrů. První parametr se vysází tučným písmem a druhý ji doplňuje klasickým textem. Pro vytvoření popisového seznamu se použije následující příkaz: \strojopis{$\backslash$popisseznam\{$\backslash$polozka\{tučný text\}\{klasický text\}}.\\[1cm]
	\novastrana
	Praktický příklad:\\
	\strojopis{
		$\backslash$popisseznam\{\\
		\indent $\backslash$polozka \{základní deska\}\{základní hardware většiny počítačů\}\\
		\indent $\backslash$polozka \{grafická karta\}\{součást počítače, jejímž úkolem je vytvářet \\
		\indent grafický výstup na monitoru\}\\
		\indent $\backslash$polozka \{procesor\}\{základní elektronická součást, která umí vykonávat\\
		\indent strojové instrukce\}\\
		\}\\[1cm]
	}
	Po vysázení:
	\popisseznam{
		\polozka {základní deska}{základní hardware většiny počítačů}
		\polozka {grafická karta}{součást počítače, jejímž úkolem je vytvářet grafický výstup na monitoru}
		\polozka {procesor}{základní elektronická součást, která umí vykonávat strojové instrukce}
	}
	
	\kapitola{Stromová struktura}
	Pro tvorbu stromové struktury slouží příkaz $\backslash$strom\{\}. Každá úroveň stromu se napíše do hranatých závorek a každá další se vnoří opět do hranatých závorek následujícím způsobem:\\
	\strojopis{
		$\backslash$strom\{\\
		\indent $[$1. úroveň\\
		\indent \indent $[$1.1 úroveň$]$\\
		\indent \indent $[$1.2 úroveň\\
		\indent \indent \indent $[$1.2.1 úroveň$]$\\
		\indent \indent $]$\\
		\indent $]$\\
		\}\\[1cm]
	}
	Praktický příklad: \\
	\strojopis{
		$\backslash$strom\{\\
		\indent$[$Strom\\
		\indent \indent $[$Větvička$]$\\
		\indent \indent $[$Lísteček\\
		\indent \indent \indent $[$Zelený$]$\\
		\indent \indent \indent $[$Oranžový$]$\\
		\indent \indent \indent $[$Žlutý$]$\\
		\indent \indent $]$\\
		\indent $]$\\
		\}
	}\\[1cm]
	Po vysázení:\\[5mm]
	\strom{
		[Strom
		[Větvička]
		[Lísteček
		[Zelený]
		[Oranžový]
		[Žlutý]
		]
		]
	}
	
	\kapitola{Tabulky}
	\podkapitola{Klasická tabulka}
	Příkaz:\\
	\strojopis{
		$\backslash$tab\{nazevTabulky\}\{Popisek k tabulce\}\{|c|c|\}\{\\
		\indent $\backslash$cara\\
		\indent prvek \& prvek\\
		\indent $\backslash$cara\\
		\indent prvek \& prvek\\
		\indent $\backslash$cara\\
		\}	
	}\\
	Pro tvorbu libovolné tabulky se do prvního povinného parametru příkazu \strojopis{$\backslash$tab} vloží vlastní jedinečný název tabulky. Pomocí tohoto názvu se bude možno v textu odhazovat na tabulku díky příkazu \strojopis{$\backslash$viztab\{nazevTabulky\}}, umístěnému bezprostředně do textu. Druhý parametr určuje popisek, jež se zobrazí pod tabulkou.\\
	Do třetího povinného parametru se vkládají následující hodnoty. Písmena \kurziva{c}, \kurziva{r}, \kurziva{l} znázorňují umístění textu v jednotlivých sloupcích tabulky, kde \kurziva{c} značí zarovnání na střed, \kurziva{r}~zarovnání vpravo a \kurziva{l} zarovnání vlevo. Pomocí svislice | se určí, mezi kterými sloupci bude vysázena vertikální čára.\\
	Do čtvrtého povinného parametru se již vloží samotná tabulka. Jednotlivé buňky se od sebe oddělí pomocí znaku ampersand \kurziva{\&} a celý řádek buněk se zakončí dvěma zpětnými lomítky \kurziva{$\backslash \backslash$}, čímž bude ihned možno začít nový řádek buněk. Pro vytvoření horizontální čáry mezi řádky se použije příkaz \strojopis{$\backslash$cara}.\\[1cm]
	\novastrana
	Praktický příklad:\\
	\strojopis{
		$\backslash$tab\{knihy\}\{Seznam knih a autorů\}\{|c|c|\}\{\\
		\indent $\backslash$cara\\
		\indent kniha \& autor\\
		\indent $\backslash$cara\\
		\indent 1984 \& George Orwell\\
		\indent Farma zvířat \& George Orwell\\
		\indent $\backslash$cara\\
		\}\\[1cm]
	}
	Po vysázení: (\viztab{knihy})
	\tab{knihy}{Seznam knih a autorů}{|c|c|}{
		\cara
		kniha & autor\\
		\cara
		1984 & George Orwell\\
		Farma zvířat & George Orwell\\
		\cara
	}
	
	\podkapitola{Profesionální tabulka}
	Profesionální tabulka se hodí pro dokumentace a od obyčejné liší tím, že obsahuje defaultně tři hlavní vodorovné čáry, kde dvě z nich jsou hlavní (čáry označující začátek a~konec jsou širší).  Pro sazbu se použije příkaz \strojopis{$\backslash$tabulka} s povinnými parametry. Na tabulku je též možno odkazovat v textu stejným způsobem jako u klasické tabulky a to pomocí příkazu \strojopis{$\backslash$viztab\{nazevTabulky\}}.\\[1cm]
	Příkaz:\\
	\strojopis{
		$\backslash$tabulka\{nazevTabulky\}\{Popisek k tabulce\}\\
		\{prvek \& prvek$\backslash \backslash$\}\\
		\{prvek \& prvek$\backslash \backslash$\\
		\indent prvek \& prvek\}\\
	}
	\\[1cm]
	Praktický příklad:\\
	\strojopis{
		$\backslash$tabulka\{latky\}\{Látky a jejich charakteristika\}\\
		\{Látka \& Skupenství \& Popis$\backslash \backslash$\}\\
		\{Voda \& kapalina \& bez chuti, bez zápachu$\backslash \backslash$\\
		Brom \& kapalina \& červenohnědý, dýmavý, vysoce toxický$\backslash \backslash$\}\\
	}
	\\[1cm]
	Po vysázení: (\viztab{latky})
	\tabulka{latky}{Látky a jejich charakteristika}
	{Látka & Skupenství & Popis\\}
	{Voda & kapalina & bez chuti, bez zápachu\\
		Brom & kapalina & červenohnědý, dýmavý, vysoce toxický\\}
	
	\kapitola{Obrázek}
	Obrázek se vloží do složky s názvem \kurziva{images} umístěné ve složce s dokumentem. Pro vložení obrázku se použije příkaz \strojopis{$\backslash$obrazek\{nazevObrazku\}\{popisek pod obrázkem\}\\
		\{šířka obrázku s jednotkou\}\{název souboru\}}.\\
	Na obrázek je samozřejmě stejně jako na tabulku či citaci možno odkazovat v textu. Použije se příkaz \strojopis{$\backslash$viz\{nazevObrazku\}}, který vytvoří odkaz na obrázek ve formátu, viz~obr.~1.1.
	\\[1cm]
	Praktický příklad:\\
	\strojopis{$\backslash$obrazek\{logoSkoly\}\{Logo školy\}\{5cm\}\{logo\}}
	\\[1cm]
	Po vysázení: (\viz{logoSkoly})
	\obrazek{logoSkoly}{Logo školy}{5cm}{logo}
	
	\kapitola{Matematický zápis}
	Matematický zápis se píše mezi dva znaky dolarů \$ \$. Zápis na více řádků se provede příkazem \strojopis{$\backslash$vzorec\{ \}}, kde se do povinného parametru vepíše daný vzorec. Každý řádek se ukončí klasicky pomocí dvou zpětných lomítek $\backslash \backslash$.  Je-li potřeba zarovnat jednotlivé řádky podle kteréhokoli znaku, na každém řádku se vloží mezi dva ampersandy (př. \&=\&).\\
	Pro automatickou velikost závorek, jež se přizpůsobí velikosti vzorce, se pro levou závorku použije \strojopis{$\backslash$leva(} a pro pravou závorku \strojopis{$\backslash$prava)}. Závorky lze do sebe vnořovat.\\[1cm]
	Zlomek: \strojopis{$\backslash$zlomek\{čitatel\}\{jmenovatel\}}\\
	Dolní index: \strojopis{\_\{dolní index\}}\\
	Horní index: \strojopis{$\textasciicircum$\{horní index\}}\\
	Matice (prvky oddělit znakem \& a řádky $\backslash \backslash$): \strojopis{$\backslash$matice\{\}}\\
	Šipka směřující vpravo: \strojopis{$\backslash$sipka}\\
	Vložení textu do vzorce: \strojopis{$\backslash$text\{text\}}\\[1cm]
	Další možnosti zápisů:\\
	Odmocnina: \strojopis{$\backslash$sqrt[n]\{vzorec pod n-tou odmocninou\}\\}
	Suma: \strojopis{$\backslash$sum\_\{spodní index\}$\textasciicircum$\{horní index\}}\\
	Limita: \strojopis{$\backslash$lim\_\{definovaní meze\}\{vzorec pro limitu\}}\\
	Integrál: \strojopis{$\backslash$int\_od$\textasciicircum$do}\\[1cm]
	\novastrana
	Praktický příklad:\\
	\strojopis{
		$\backslash$vzorec\{\\
		1+2\&=\&3$\backslash \backslash$\\
		B\_\{1\}\&=\&$\backslash$sum\_\{i=1\}$\textasciicircum$N m\_i $\backslash$leva[ x$\textasciicircum$i\_2 $\backslash$leva( $\backslash$omega\_1 x$\textasciicircum$i\_2 - $\backslash$omega\_2\\ x$\textasciicircum$i\_1 $\backslash$prava) - x$\textasciicircum$i\_3 $\backslash$leva( $\backslash$omega\_3 x$\textasciicircum$i\_1 - $\backslash$omega\_1 x$\textasciicircum$i\_3 $\backslash$prava) \\
		$\backslash$prava]$\backslash \backslash$ \}	
	}\\[1cm]
	Po vysázení:
	\vzorec{
		1+2&=&3\\
		B_{1}&=&\sum_{i=1}^N m_i \leva[ x^i_2 \leva( \omega_1 x^i_2 - \omega_2 x^i_1 \prava) - x^i_3 \leva( \omega_3 x^i_1 - \omega_1 x^i_3 \prava) \prava]\\ }
	Další funkce a symboly naleznete v přílohách.
	
	\kapitola{Přílohy}
	Příkaz \strojopis{$\backslash$prilohy} vytvoří přílohy označené písmeny dle abecedy. Do povinného parametru se vkládají přílohy stejným způsobem, jakoby šlo o kapitoly (viz 4.3 Kapitoly a podkapitoly).\\
	\strojopis{
		$\backslash$prilohy\{\\
		\indent $\backslash$kapitola\{název přílohy1\}\\
		\indent $\backslash$kapitola\{název přílohy2\}\\
		\} 
	}
	
	\kapitola{Citace a použitá literatura}
	Pro dokumentaci je použita citace dle ISO 690 formou číselných odkazů. Pro uvedení odkazu na citaci je potřeba vložit bezprostředně za text příkaz \strojopis{$\backslash$citace\{nazevCitace\}}. Do povinného parametru se uvede libovolně zvolený název citace. Tento název bude jednoznačnou identifikací citace pro celý dokument a bude se na ni odkazovat v seznamu literatury.\\
	Na konci dokumentu je uveden příkaz \strojopis{$\backslash$literatura\{seznam literatury\}}, jež započne seznam použitých zdrojů (citací, literatury). V povinném parametru se definuje konkrétní seznam, který se vytvoří z následujících příkazů.\\[1cm]
	Citace pro knihu:\\
	\strojopis{
		$\backslash$kniha\{nazevCitace\}\{Příjmení autora\}\{Jméno autora\}\{Název knihy\}\\
		\{Místo vydání\}\{Nakladatelství\}\{Rok\}\{ISBN\}
	}
	\\[1cm]
	Citace pro kvalifikační práci:\\
	\strojopis{
		$\backslash$kvalifikacniprace\{nazevCitace\}\{Příjmení autora\}\{Jméno autora\}\\
		\{Název práce\}\{Místo\}\{Rok\}\{Druh práce\}\{Univerzita, Fakulta, Katedra\}\\
		\{Vedoucí diplomové práce jméno\}
	}\\
	Vysvětlivky: druh práce – např: Bakalářská práce, Diplomová práce
	\\[1cm]
	Citace pro URL adresu:\\
	\strojopis{
		$\backslash$url\{nazevCitace\}\{Název stránek\}\{Titulek\}\{Stránky\}\{Rok\}\{Datum\}\{URL odkaz\}	
	}\\
	Vysvětlivky: datum -- datum, kdy se cituje; použije se libovolný datum ve formátu RRRR-MM-DD nebo se využije příkaz \strojopis{$\backslash$dnes}
	\\[1cm]
	Praktický příklad:\\
	\strojopis{
		$\backslash$literatura$\{$\\
		\indent $\backslash$kniha$\{$princ$\}\{$Saint-Exupéry$\}\{$Antoine~de$\}\{$Malý princ$\}\{$Praha$\}$\\
		\indent $\{$Státní nakladatelství dětské knihy$\}\{$1966$\}\{$587665858$\}$\\
		\indent $\backslash$kniha$\{$zaklety$\}\{$Niffenegger$\}\{$Audrey$\}\{$Zakletý v čase$\}\{$Praha$\}$\\
		\indent $\{$Argo$\}\{$2009$\}\{$978-80-257-0222-2$\}$\\
		\indent $\backslash$kvalifikacniprace$\{$diplomka$\}\{$Knotek$\}\{$Pavel$\}\{$Kultura jako péče o duši$\}$\\
		\indent $\{$Praha$\}\{$1999$\}\{$Diplomová práce$\}\{$Karlova Univerzita, Filozofická fakulta, \\
		\indent Katedra andragogiky$\}\{$Vedoucí diplomové práce Zdeněk Kratochvíl$\}$\\
		\indent $\backslash$url$\{$adresa$\}\{$Westcom$\}\{$O nás$\}\{$Webnode.cz$\}\{$2018$\}\{\backslash$dnes$\}$\\
		\indent $\{$http://www.webnode.cz/o-nas/$\}$\\
		$\}$
	}\\[1cm]
	Po vysázení: (\viz{sazba06})
	\obrazek{sazba06}{Literatura}{12cm}{sazba06}
	
	\kapitola{Seznam příkazů}
	\begin{table}[h]
		\centering
		\begin{tabular}{ll}	
			\toprule[1.5pt]
			Příkaz & Význam\\
			\midrule
			$\backslash$student$\{$Jméno Příjmení$\}$ & Jméno a příjmení autora\\
			$\backslash$trida$\{$Třída$\}$ & Třída\\
			$\backslash$obor$\{$Kód oboru Název oboru$\}$ & Kód a název oboru\\
			$\backslash$bydliste$\{$Adresa bydliště$\}$ & Bydliště autora\\
			$\backslash$datumNarozeni$\{$Datum narození$\}$ & Datum narození autora\\
			$\backslash$vedouci$\{$Titul, jméno, příjmení$\}$ & Titul, jméno a příjmení vedoucí/ho\\
			$\backslash$nazevPrace$\{$Název práce$\}$ & Název práce\\
			$\backslash$cisloPrace$\{$Číslo$\}$ & Číslo tématu práce\\
			$\backslash$skolniRok$\{$Školní rok$\}$ & Školní rok\\
			$\backslash$reditel$\{$Titul, jméno a příjmení$\}$ & Titul, jméno a příjmení ředitele školy\\
			\bottomrule[1,5pt]
		\end{tabular}
	\end{table}
	
	\begin{table}
		\centering
		\begin{tabular}{ll}	
			\toprule[1.5pt]
			Příkaz & Význam\\
			\midrule
			$\backslash$zacatek & Začátek dokumentu\\
			$\backslash$konec & Konec dokumentu\\
			$\backslash$anotace & Anotace v českém jazyce\\
			$\backslash$annotation & Anotace v anglickém jazyce\\
			$\backslash$podekovani & Poděkování\\
			$\backslash$obsah & Vygeneruje obsah\\
			$\backslash$seznamTabulek & Vygeneruje seznam tabulek\\
			$\backslash$seznamObrazku & Vygeneruje seznam obrázků\\
			$\backslash$titulniStrana & Vygeneruje titulní stranu\\
			$\backslash$zadani$\{$termín odevzdání$\}\{$zadání$\}$ & Vygeneruje zadání\\
			$\{$originalita$\}\{$funkčnost$\}\{$náklady$\}$\\
			$\{$majetek$\}\{$datum podepsání$\}$\\
			$\backslash$licencniSmlouva$\{$Datum odevzdání$\}$ & Vygeneruje licenční smlouvu\\
			$\backslash$kapitola$\{$název$\}$ & Vytvoří kapitolu\\
			$\backslash$podkapitola$\{$název$\}$ & Vytvoří subkapitolu\\	
			$\backslash$podpodkapitola$\{$název$\}$ & Vytvoří subsubkapitolu\\
			$\backslash$podpodpodkapitola$\{$název$\}$ & Vytvoří subsubsubkapitolu\\
			$\backslash$novastrana & Vynucení nové strany\\
			$\backslash$uv$\{$text$\}$ & Vypíše text v uvozovkách\\
			$\backslash$strojopis$\{$text$\}$ & Styl textu: strojopis\\
			$\backslash$kurziva$\{$text$\}$ & Styl textu: kurzíva\\
			$\backslash$tucne$\{$text$\}$ & Styl textu: tučné\\
			$\backslash$kapitalky$\{$text$\}$ & Styl textu: kapitálky\\
			$\backslash$zvyraznit$\{$text$\}$ & Styl textu: zvýrazněné\\
			\bottomrule[1,5pt]
		\end{tabular}
	\end{table}
	
	\begin{table}
		\centering
		\begin{tabular}{ll}	
			\toprule[1.5pt]
			Příkaz & Význam\\
			\midrule
			$\backslash$cislseznam$\{ \backslash$bod text$\}$ & Číslovaný seznam\\
			$\backslash$bodseznam$\{ \backslash$bod text$\}$ & Nečíslovaný seznam\\
			$\backslash$popisseznam$\{ \backslash$polozka$\{$tučný text$\}$ & Popisový seznam\\
			$\{$klasický text$\}\}$\\
			$\backslash$strom$\{\}$ & Stromová struktura\\
			$\backslash$tab$\{$nazevTabulky$\}\{$Popisek k tabulce$\}$ & Klasická tabulka\\
			$\{|$c$|$c$|\}\{ \backslash$cara prvek \& prvek$\backslash \backslash$ $\backslash$cara$\}$\\
			$\backslash$tabulka$\{$nazevTabulky$\}\{$Popisek k tabulce$\}$ & Profesionální tabulka\\
			$\{$prvek \& prvek$\backslash \backslash \}$$\{$prvek \& prvek $\backslash \backslash \}$\\
			$\backslash$viztab$\{$nazevTabulky$\}$ & Odkaz na tabulku v textu\\
			$\backslash$obrazek$\{$nazevObrazku$\}\{$popisek$\}\{$šířka$\}$ & Vložení obrázku\\
			$\{$soubor$\}$\\
			$\backslash$viz$\{$nazevObrazku$\}$ & Odkaz na obrázek v textu\\
			$\backslash$vzorec$\{ \}$ & Matematický zápis i na více řádků\\
			$\backslash$zlomek$\{$čitatel$\}\{$jmenovatel$\}$ & Zlomek\\
			$\backslash$matice$\{\}$ & Matice\\
			$\backslash$sipka & Šipka vpravo\\
			$\backslash$leva$($ & Automatická velikost levé závorky\\
			$\backslash$prava$)$ & Automatická velikost pravé závorky\\
			$\backslash$text$\{$text$\}$ & Vložení textu do vzorce\\
			$\backslash$prilohy$\{\backslash$kapitola$\{$Příloha$\}$ $\backslash$kapitola & Vytvoří přílohy\\
			$\{$Příloha 2$\}\}$\\
			\bottomrule[1,5pt]
		\end{tabular}
	\end{table}
	
	\begin{table}
		\centering
		\begin{tabular}{ll}	
			\toprule[1.5pt]
			Příkaz & Význam\\
			\midrule
			$\backslash$literatura$\{$seznam literatury$\}$ & Obsahuje seznam literatury\\
			$\backslash$citace$\{$nazevCitace$\}$ & Odkaz na literaturu v textu\\
			$\backslash$kniha$\{$nazevCitace$\}\{$Příjmení autora$\}$ & Citace pro knihu\\
			$\{$Jméno autora$\}\{$Název knihy$\}\{$Místo vydání$\}$\\
			$\{$Nakladatelství$\}\{$Rok$\}\{$ISBN$\}$\\
			$\backslash$kvalifikacniprace$\{$nazevCitace$\}$ & Citace pro kvalifikační práci\\
			$\{$Příjmení autora$\}\{$Jméno autora$\}$\\
			$\{$Název práce$\}\{$Místo$\}\{$Rok$\}$\\
			$\{$Druh práce$\}\{$Univerzita, Fakulta, Katedra$\}$\\
			$\{$Vedoucí diplomové práce jméno$\}$\\
			$\backslash$url$\{$nazevCitace$\}\{$Název stránek$\}$ & Citace pro URL adresu\\
			$\{$Titulek$\}\{$Stránky$\}\{$rok$\}\{$datum$\}\{$URL odkaz$\}$\\
			$\backslash$dnes & Dnešní datum pro citaci URL\\
			$\backslash$sqrt$[$n$]\{$vzorec pod n-tou odmocninou$\}$ & Odmocnina\\
			$\backslash$sum\_$\{$spodní index$\}$\textasciicircum $\{$horní index$\}$ & Suma\\	
			$\backslash$lim\_$\{$definovaní meze$\}\{$vzorec pro limitu$\}$ & Limita\\
			$\backslash$int\_od\textasciicircum do & Integrál\\
			\_{dolní index} & Dolní index\\
			\textasciicircum{horní index} & Horní index\\
			\$ vzorec \$  & Matematický zápis na jeden řádek\\
			$\backslash$\$ & Zpětné lomítko před speciální znaky\\
			$\sim$ & Nedělitelná mezera\\
			$\backslash \backslash$ & Zalamování textu\\
			\% & Komentář\\
			\bottomrule[1,5pt]
		\end{tabular}
	\end{table}
	
	\begin{table}
		\centering
		\begin{tabular}{rlrlrlrl}
			\hline
			arccos & $\backslash arccos$ & arcsin & $\backslash$ arcsin & arctan & $\backslash arctan$ & arg & $\backslash arg$\\
			cos & $\backslash cos$ & cosh & $\backslash cosh$ & cot & $\backslash cot$ & coth & $\backslash coth$\\
			csc & $\backslash csc$ & deg & $\backslash deg$ & det & $\backslash det$ & dim & $\backslash dim$ \\
			exp & $\backslash exp$ & gcd & $\backslash gcd$ & hom & $\backslash hom$ & inf &  $\backslash inf$\\
			ker & $\backslash exp$ & lg & $\backslash lg$ & lim & $\backslash lim$ & lim inf & $\backslash liminf$\\
			lim sup & $\backslash limsup$ & ln & $\backslash ln$ & log & $\backslash log$ & max & $\backslash max$\\
			min & $\backslash min$ & pr & $\backslash p$r & sec & $\backslash sec$ & sin & $\backslash sin$\\
			sinh & $\backslash sinh$ & sup & $\backslash sup$ & tan & $\backslash tan$ & tanh & $\backslash tanh$\\
			\hline
		\end{tabular}
	\end{table}

	\begin{table}
		\centering
		\begin{tabular}{rlrlrlrlrl}
			\hline
			$\alpha$ & $\backslash$alpha & $\beta$ & $\backslash$beta & $\gamma$ & $\backslash$gamma & $\delta$ & $\backslash$delta & $\epsilon$ & $\backslash$epsilon\\
			$\varepsilon$ & $\backslash$varepsilon & $\zeta$ & $\backslash$zeta & $\eta$ & $\backslash$eta & $\theta$ & $\backslash$theta & $\vartheta$ & $\backslash$vartheta\\
			$\iota$ & $\backslash$iota & $\kappa$ & $\backslash$kappa & $\lambda$ & $\backslash$lambda & $\mu$ & $\backslash$mu & $\nu$ & $\backslash$nu\\ 
			$\xi$ & $\backslash$xi & $o$ & o & $\pi$ & $\backslash$pi & $\varpi$ & $\backslash$varpi & $\rho$ & $\backslash$rho\\
			$\varrho$ & $\backslash$varrho & $\sigma$ & $\backslash$sigma & $\varsigma$ & $\backslash$varsigma & $\tau$ & $\backslash$tau & $\upsilon$ & $\backslash$upsilon\\
			$\phi$ & $\backslash$phi & $\varphi$ & $\backslash$varphi & $\chi$ & $\backslash$chi & $\psi$ & $\backslash$psi & $\omega$ & $\backslash$omega\\
			\hline
		\end{tabular}
	\end{table}

	\begin{table}
		\centering
		\begin{tabular}{rlrlrlrl}
			\hline
			$^\circ C$ & \textasciicircum$\backslash$circ C & $\infty$ & $\backslash$infty & $\partial$ & $\backslash$partial & $\triangle$ & $\backslash$triangle\\
			$\rightarrow$ & $\backslash$rightarrow & $\leq$ & $\backslash$leq & $\geq$ & $\backslash$geq & $\neq$ & $\backslash$neq\\
			$\in$ & $\backslash$in & $\ll$ & $\backslash$ll & $\gg$ & $\backslash$gg & $\doteq$ & $\backslash$doteq\\
			$\pm$ & $\backslash$pm & $\div$ & $\backslash$div & $\approx$ & $\backslash$approx & $\backslash$ & $\backslash$backslash\\
			\hline
		\end{tabular}
	\end{table}
	
\end{document}

