\documentclass[a4paper, 12pt]{report}
\usepackage{monapack}

\student{Václav Zíka}
\trida{B4.I}
\obor{18-20-M/01 Informační technologie}
\bydliste{Na Spravedlnosti 974/36, Písek 39701, Česká republika}
\datumNarozeni{29. 12. 2005}
\vedouci{Mgr. Milan Janoušek}
\nazevPrace{Elektromechanická hra -> samořídící\,šachovnice}
\cisloPrace{4.}
\skolniRok{2024/2025}
\reditel{Ing. Jiří Uhlík}

\zacatek
	
	\titulniStrana

	\zadani{31. 3. 2025}
	{
		\bod
		Proveďte teoretický úvod k problematice samořídící šachovnice řešící realizaci šachové desky
ovládané mikrokontrolérem, uživatelské ovládání šachovnice, vhodné komponenty (součástky,
šachové figurky, hrací pole).
		\bod
		Realizujete vlastní řešení v následujících bodech
		\cislseznam{
			\bod
			Navrhněte všechna potřebná schémata pro realizaci samořídící šachovnice.
			\bod
			Vyberte vhodný mikrokontrolér pro řízení šachovnice.
			\bod
			Vytvořte program pro zvolený mikrořadič.
			\bod
			Vyřešte pohyb figurek po šachovnici a sestrojte řešení.
			\bod
			Navrhněte způsob detekce obsazených polí na šachovnici.
			\bod
			Vyřešte komunikaci mezi řízením šachovnice a systémem, který bude simulovat tahy protihráče (např. serverem, algoritmem...).
			\bod
			Navrhněte a potřebnými součástkami osaďte desku šachovnice.
			\bod
			Realizujte konstrukci šachovnice.
			\bod
			Výsledné řešení prakticky ověřte.
		}
		\bod
		Zpracujte dokumentaci dle metodického návrhu a ppt prezentaci pro účely obhajoby.
		\bod
		Propagujte výsledky své práce - např. vyhotovením posteru, účastí na SOČ, zhotovení informační
		www stránky, natočení promo videa apod.
	}
	{
		\bod
		výběr řídící jednotky
		\bod
		mechanické provedení šachovnice s políčky a figurkami
		\bod
		obsazenost detekujících se políček
		\bod
		možnost hry s protihráčem
		\bod
		možnosti programového kódu
	}
	{
		\bod
		pohyb figurek na hracím poli
		\bod
		detekce obsazenosti políček
		\bod
		algoritmus vytvořeného programu
		\bod
		použitelnost k šachové hře
	}
	{žák}
	{žáka}
	{15. 11. 2024}
	
	\anotace 
	Maturitní práce se zabývala tvorbou samořídící šachovnice, jejíž cílem bylo kompletně simulovat protihráče.
	Ať už se jedná o vymýšlení protitahu či o samotný manuální posun figurky. Šachovnice je ovládána mikrokontrolérem
	Arduino~UNO. Pomocí magnetických spínačů umístěných na PCB desce detekujeme pozici figurek na hracím poli. Tyto informace
	následně zpracováváme Arduinem a skrze dva krokové motory a elektromagnet realizujeme tahy figurek. Pro zjištění
	ideálního příštího tahu využíváme Minimax algoritmus, jenž je schopný zevaluovat danou situaci a skrze programovou
	logiku udat příkazy pro pohyb figurek. Celá samořídící šachovnice je vyrobena z dřevěné konstrukce, do které je
	umístěna deska plošného spoje s elektronikou. V rámci projektu jsem si pomocí 3D tisku zhotovil magnetické šachové figurky. 
	Šachovnice také umožňuje zvolení si obtížnosti a volbu hry buď za černé či bílé.

	
	
	\annotation
	Aj...

	
	\podekovani
	 Chtěl bych poděkovat panu Mgr. Janouškovi za podporu a vedení práce. Dále elektrotechnikům,
	 co mi pomáhali s výrobou desky, Jířímu Kutilovi, dále mojí rodině, přítelkyni, bráchovi a ségře,
	 Marečkovi Vejmelkovi a především světovému míru a mému perpetum mobile Tadeášovi H(a)nusovi.
	
		
	\obsah
	
	\kapitola{Úvod}
		Nápad pro automatickou šachovnici jsem dostal při hledání nového projektu,
		který bych si doma zvládl sestrojit. Chtěl jsem, aby projekt obsahoval, jak 
		část softwarovou, kterou jsem se do té doby primárně zabýval, ale také část 
		mechanickou, kterou by bylo nutné vyrobit. 

		Napadlo mě vytvořit nějakou deskovou hru pro více hračů. V té době jsem měl
		ve velké oblibě šach a tím vznikla idea automatická šachovnice. Začal jsem nákresem
		na papír a tím jsem získal základní představu o projektu. Šachovnice se skládá ze tří
		hlavní částí:

		\popisseznam{
			\polozka {Konstrukční část} {
				Tou jest samotná dřevěná konstrukce šachovnice, do které bude nutné umístít mechanismus
				pro pohybování s figurky. V rámci toho také její opracování a  nadesignování projektu. 
				Tvorba šachových políček a rozhraní pro jednoduché ovládání.
			}
			\polozka {Elektrotechnická část}{
				Ta obsahuje systém pro detekci figurek na šachovnici. Ten by se dal vytvořit pomocí mnoha
				způsobů, ale v práci budu popisovat řešení pomocí desky plošného spoje. Dále vytvoření pohybové 
				soustavy pomocí krokových motorů a umístění elektromagnetu, jenž bude s figurky pohybovat. Posledním
				krokem je sestrojení systému, jenž umožní uživateli variabilovat hru, pomocí volby barvy a obtížnosti.
			}
			\polozka {Softwarová část}{
				Z hlediska softwaru je nutné vytvořit kód, který spojí všechny části dohromady. Je nutné transformovat
				signály, tak aby s nimi bylo možné pracovat. Z hlediska kódu je nutné vytvořit rozhraní pro komunikaci
				mezi mikrokontrolérem a součástky. Je zapotřebí umožnit evaluaci dat o pozicích figurek a předat je
				algoritmu, který nám bude schopen vymyslet další tah. Údáje o dalším tahu ovšem musíme opět transformovat,
				tak aby z něj byli čitelné pokyny pro krokové motory. Nesmíme opomenou ani řídící signály a minimalizovat
				zpoždění a zatížení na hardware.
			}
		}


	
	\kapitola{Řídící jednotka}
		
	\kapitola{Šachové herní pole}	
	
	\kapitola{Programový kód}
		
	\kapitola{Konstrukční provedení}
	
	\kapitola{Šachové figurky}
			
	\kapitola{Závěr}

	\seznamTabulek
	
	\seznamObrazku
	
	\prilohy{
		\kapitola{Příloha}
	}
	
	\literatura{
		
		\kniha{nazevCitace}{Příjmení autora}{Jméno autora}{Název knihy}{Místo vydání}{Nakladatelství}{Rok}{ISBN}
		
		\kvalifikacniprace{nazevCitace}{Příjmení autora}{Jméno autora}{Název práce}{Místo}{Rok}{Druh práce}{Univerzita, Fakulta, Katedra}{Vedoucí diplomové práce jméno}
		
		\url{nazevCitace}{Název stránek}{Titulek}{Stránky}{rok}{datum}{URL odkaz}
		
	}
	
\konec