\documentclass[a4paper, 12pt]{report}
\usepackage{monapack}

\student{Václav Zíka}
\trida{B4.I}
\obor{18-20-M/01 Informační technologie}
\bydliste{Na Spravedlnosti 974/36, Písek 39701, Česká republika}
\datumNarozeni{29. 12. 2005}
\vedouci{Mgr. Milan Janoušek}
\nazevPrace{Elektromechanická hra -> samořídící\,šachovnice}
\cisloPrace{4.}
\skolniRok{2024/2025}
\reditel{Ing. Jiří Uhlík}

\zacatek
	
	\titulniStrana

	\zadani{31. 3. 2025}
	{
		\bod
		Proveďte teoretický úvod k problematice samořídící šachovnice řešící realizaci šachové desky
ovládané mikrokontrolérem, uživatelské ovládání šachovnice, vhodné komponenty (součástky,
šachové figurky, hrací pole).
		\bod
		Realizujete vlastní řešení v následujících bodech
		\cislseznam{
			\bod
			Navrhněte všechna potřebná schémata pro realizaci samořídící šachovnice.
			\bod
			Vyberte vhodný mikrokontrolér pro řízení šachovnice.
			\bod
			Vytvořte program pro zvolený mikrořadič.
			\bod
			Vyřešte pohyb figurek po šachovnici a sestrojte řešení.
			\bod
			Navrhněte způsob detekce obsazených polí na šachovnici.
			\bod
			Vyřešte komunikaci mezi řízením šachovnice a systémem, který bude simulovat tahy protihráče (např. serverem, algoritmem...).
			\bod
			Navrhněte a potřebnými součástkami osaďte desku šachovnice.
			\bod
			Realizujte konstrukci šachovnice.
			\bod
			Výsledné řešení prakticky ověřte.
		}
		\bod
		Zpracujte dokumentaci dle metodického návrhu a ppt prezentaci pro účely obhajoby.
		\bod
		Propagujte výsledky své práce - např. vyhotovením posteru, účastí na SOČ, zhotovení informační
		www stránky, natočení promo videa apod.
	}
	{
		\bod
		výběr řídící jednotky
		\bod
		mechanické provedení šachovnice s políčky a figurkami
		\bod
		obsazenost detekujících se políček
		\bod
		možnost hry s protihráčem
		\bod
		možnosti programového kódu
	}
	{
		\bod
		pohyb figurek na hracím poli
		\bod
		detekce obsazenosti políček
		\bod
		algoritmus vytvořeného programu
		\bod
		použitelnost k šachové hře
	}
	{žák}
	{žáka}
	{15. 11. 2024}
	
	\anotace 
	Text

	\annotation
	Text
	
	\podekovani
	Text
		
	\obsah
	
	\kapitola{Úvod}
	\kapitola{Vlastní text práce}
		rozvedený do jednotlivých kapitol a subkapitol
		\podkapitola{Subkapitola}
			\podpodkapitola{Subsubkapitola}
			
	\kapitola{Závěr}

	\seznamTabulek
	
	\seznamObrazku
	
	\prilohy{
		\kapitola{Příloha}
	}
	
	\literatura{
		
		\kniha{nazevCitace}{Příjmení autora}{Jméno autora}{Název knihy}{Místo vydání}{Nakladatelství}{Rok}{ISBN}
		
		\kvalifikacniprace{nazevCitace}{Příjmení autora}{Jméno autora}{Název práce}{Místo}{Rok}{Druh práce}{Univerzita, Fakulta, Katedra}{Vedoucí diplomové práce jméno}
		
		\url{nazevCitace}{Název stránek}{Titulek}{Stránky}{rok}{datum}{URL odkaz}
		
	}
	
\konec